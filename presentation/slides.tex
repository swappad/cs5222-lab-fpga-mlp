\documentclass{beamer}

\usepackage[utf8]{inputenc}
\usepackage{amsmath}
\usepackage[binary-units=true]{siunitx}
\usepackage{enumitem}
\usepackage{graphicx}
\usepackage{caption}
\usepackage{subcaption}
\usepackage{amssymb}
\usepackage{multirow}
\usepackage{multicol}
\usepackage{listings}
\usepackage{tikz}
\usetikzlibrary{positioning}

\lstset{
            language=bash,
            basicstyle=\ttfamily \small
        }


%Information to be included in the title page:
\title{Part 3 Open End Optimization}
\author{Luca Krueger}
\institute{CS5222 Project 2}
\date{\today}

\usetheme{Berlin}
\usecolortheme{beaver}


\begin{document}

%\frame{\titlepage}

\begin{frame}
\frametitle{Learning Algorithm}
\tikzset{%
   neuron missing/.style={
    draw=none, 
    scale=2,
    text height=0.333cm,
    execute at begin node=\color{black}$\vdots$
  },
}

\begin{tikzpicture}[scale=0.9, >=stealth]

\foreach \m/\l [count=\y] in {1,2,3}
{
 \node [circle,fill=green!50,minimum size=0.5cm] (input-\m) at (0,2.5-\y) {};
}
\foreach \m/\l [count=\y] in {4}
{
 \node [circle,fill=green!50,minimum size=0.5cm ] (input-\m) at (0,-2.5) {};
}
 
 \node [neuron missing]  at (0,-1.5) {};
 

\foreach \m [count=\y] in {1}
  \node [circle,fill=red!50,minimum size=0.5cm ] (hidden-\m) at (2,0.75) {};
  
\foreach \m [count=\y] in {2}
  \node [circle,fill=red!50,minimum size=0.5cm ] (hidden-\m) at (2,-1.85) {};
  
 \node [neuron missing]  at (2,-0.3) {};


%\foreach \m [count=\y] in {1}
%  \node [circle,fill=blue!50,minimum size=1cm ] (output-\m) at (4,1.5-\y) {};
%  
%\foreach \m [count=\y] in {2}
%  \node [circle,fill=blue!50,minimum size=1cm ] (output-\m) at (4,-0.5-\y) {};

% \node [neuron missing]  at (4,-0.4) {};

\foreach \l [count=\i] in {1,2,3,256}
  \draw [<-] (input-\i) -- ++(-1,0)
    node [above, midway] {$I_{\l}$};

\foreach \l [count=\i] in {1,10}
  \node [above] at (hidden-\i.north) {$H_{\l}$};

%\foreach \l [count=\i] in {1,16}
%  \draw [->] (output-\i) -- ++(1,0)
%    node [above, midway] {$O_{ \l}$};

\foreach \i in {1,...,4}
  \foreach \j in {1,...,2}
    \draw [->] (input-\i) -- (hidden-\j);

%\foreach \i in {1,...,2}
%  \foreach \j in {1,...,2}
%    \draw [->] (hidden-\i) -- (output-\j);

%\foreach \l [count=\x from 0] in {Input, Hidden, Ouput}
 % \node [align=center, above] at (\x*2,2) {\l \\ layer};

\end{tikzpicture}

\end{frame}

\begin{frame}
\frametitle{Hardware Improvements}
New learning algorithm does not require a transferfunction other than argmax for classification:
\begin{align*}
		\text{argmax}(x) = \text{argmax}(\text{softmax}(\text{SELU}(x))) \\
\end{align*}
	\begin{itemize}
		\item[-] replaced one-hot encoding of the output with actual output value
		\item[$\Rightarrow$] FPGA$\rightarrow$CPU transmission reduced by $90\si{\percent}$ to $1\si{\byte}$ per sample. 
		\item[-] increased \texttt{Tiling value} to $128$ samples
	\end{itemize}
\end{frame}

\begin{frame}[fragile]
	\frametitle{Speedup and Accuracy}
	\textbf{On the hardware:}
\begin{lstlisting}
FPGA accuracy: 8.15% validation error
CPU accuracy:  8.06% validation error
FPGA time: [-] 0.004762924000033308
CPU time:  0.2810866919999171
FPGA has a 59.02x speedup 
\end{lstlisting}
\textbf{In the simulation:}\\
\vspace{0.5cm}
    \begin{tabular}{ccccc}
        \multicolumn{2}{c}{Latency} & \multicolumn{2}{c}{Interval} & Pipeline\\
        \hline
        min  &   max  &   min  &   max  &   Type  \\
	    374311&  374311&  374312&  374312&   none  
    \end{tabular}


\end{frame}

\begin{frame}
\frametitle{Problems}
\begin{itemize}
	\item[-] Runtime on the hardware appears to be longer than in simulation. Why?
	\item[-] Vivado HLS $2017.1$ does not reveal, how a loop is actually pipelined. Makes it hard to find further improvements.
	\item[-] The same mistake in both hls code and testbench is almost impossible to find.
\end{itemize}

\end{frame}


\end{document}


\documentclass[../main.tex]{subfiles}

\begin{document}
\begin{enumerate}[label=(\arabic*)]
	\item After setting the \texttt{SCALE} factor in \texttt{mnist.py} to $100000$, a fixed-point validation accuracy of $19.76\%$ can be achieved.
		\lstset{
			language=bash,
			basicstyle=\ttfamily
		}
		\begin{lstlisting}
Misclassifications (float) = 13.77%
Misclassifications (fixed) = 19.76%
		\end{lstlisting}

	\item Design latencies:

\begin{table}[H]
	\centering
	\begin{tabular}{ccccc}
		\multicolumn{2}{c}{Latency} & \multicolumn{2}{c}{Interval} & Pipeline\\
		\hline
		min  &   max  &   min  &   max  &   Type  \\
		444135&  444135&  444136&  444136&   none  
	\end{tabular}
	\caption{Latencies: fixed point}
	\label{2-perf-table}
\end{table}

\item Overall design utilization:
\begin{table}[H]
	\centering
	\begin{tabular}{lcccc}
		Name      & BRAM\_18K& DSP48E&   FF   &  LUT  \\
		\hline
		DSP              &        -&    129&       -&      -\\
		Expression       &        -&      -&       0&   3993\\
		FIFO             &        -&      -&       -&      -\\
		Instance         &        0&      0&    8672&   9184\\
		Memory           &        4&      -&    8192&   2560\\
		Multiplexer      &        -&      -&       -&   8252\\
		Register         &        -&      -&    6580&     96\\
		\hline
		Total            &        4&    129&   23444&  24085\\
		Available        &      280&    220&  106400&  53200\\
		\hline
		Utilization ($\%$)  &        1&     58&      22&     45
	\end{tabular}
	\caption{Resource utilization: fixed point}
	\label{2-resources}
\end{table}

\item The measured system speedup over the CPU fixed point computation is $74.92$.
\item The measured classification on the 8k MNIST test sample is $20.96\%$.
	\begin{lstlisting}
FPGA accuracy: 20.69% validation error
CPU accuracy:  20.69% validation error
FPGA has a 74.92x speedup
	\end{lstlisting}


\item Instantiation of multipliers
	\begin{table}[H]
	\centering
	\begin{tabular}{lcccc}
		Module & Number of instances \\
		\hline
		DSP48 & 129 \\
		HW multiplier modules & 127 \\
		\hline 
		\textbf{Total} & 256
	\end{tabular}
	\caption{Utilization of multipliers: fixed point}
	\label{2-resources-arithmetic}
\end{table}


\item The pipelined inner-loop of the matrix multiplication achieves an initiation interval of a single cycle, as shown in table \ref{2-loop-latencies}.
	\begin{table}[H]
	\centering
	\begin{tabular}{l|ccccccc}
				  &    \multicolumn{2}{c}{Latency}     & Iteration&  \multicolumn{2}{c}{Initiation Interval}  & Trip &           \\
				Loop Name	 &   min  &   max  &  Latency &  achieved &   target  & Count& Pipelined \\
			  \hline
          - LOAD\_OFF\_1    &       5&       5&         2&          1&          1&     5&    yes    \\
          - LOAD\_W\_1      &     350&     350&        35&          -&          -&    10&    no     \\
           + LOAD\_W\_2     &      32&      32&         2&          1&          1&    32&    yes    \\
		   \hline
          - LT            &  443776&  443776&      3467&          -&          -&   128&    no     \\
           + LOAD\_I\_1     &    2240&    2240&        35&          -&          -&    64&    no     \\
            ++ LOAD\_I\_2   &      32&      32&         2&          1&          1&    32&    yes    \\
           + L1\_L2        &     646&     646&         8&          1&          1&   640&    yes    \\
		   \hline
           + STORE\_O\_1    &     576&     576&         9&          -&          -&    64&    no     \\
            ++ STORE\_O\_2  &       6&       6&         3&          1&          1&     5&    yes    
	\end{tabular}
	\caption{Snapshot of the loop latency table}
	\label{2-loop-latencies}
\end{table}

\item Regarding the loop latency table (table \ref{2-loop-latencies}), the loading the tiled set of input values requires $\approx 3.5x$ more cycles than the actual matrix multiplication. Hence we can say that this design is bandwidth limited by the bandwidth of the AXI port.

\end{enumerate}

\end{document}



